\section{Basic Event Scheduler Model}
\label{sec:basicEventScheduler}

This model is designed to read the initial and final time under which a set of basic events
are set to True, and then it generates an HistorySet with the full temporal profile of all
basic events.

This HistorySet is generated by colleting all the initial and final time values, by ordering 
them in ascending order, and by setting the corresponding logical values for each basic event 
along the time axis.
Note that this model performs a conversion from interval-based data to instant-based data 
(i.e., the history set). The convention is that a logical value set to 1 at a specific time 
instant implies that the basic event is actually True from the previous time instant.

All the specifications of the Basic Event Scheduler model are given in the \xmlNode{ExternalModel} block.
Inside the \xmlNode{ExternalModel} block, the XML
nodes that belong to this models are:
\begin{itemize}
  \item  \xmlNode{variables}, \xmlDesc{string, required parameter}, a list containing the names of both the input and output variables of the model
  \item  \xmlNode{BE}, \xmlDesc{string, required parameter}, the name ID of all basic events
	  \begin{itemize}
	    \item \xmlAttr{tin}, \xmlDesc{required string attribute}, name ID of the variable describing the initial time of the BE
	    \item \xmlAttr{tfin}, \xmlDesc{required string attribute}, name ID of the variable describing the final time of the BE
	  \end{itemize}
\end{itemize}

The example of \textbf{basicEventScheduler} model is provided below:
\begin{lstlisting}[style=XML]
  <Models>
    <ExternalModel name="basicEventScheduler" subType="SR2ML.basicEventScheduler">
      <variables>BE1_tin, BE2_tin, BE3_tin,
                 BE1_tfin,BE2_tfin,BE3_tfin,
                 BE1,BE2,BE3,time</variables>
      <BE tin='BE1_tin' tfin='BE1_tfin'>BE1</BE>
      <BE tin='BE2_tin' tfin='BE2_tfin'>BE2</BE>
      <BE tin='BE3_tin' tfin='BE3_tfin'>BE3</BE>
      <timeID>time</timeID>
    </ExternalModel>
  </Models>
\end{lstlisting}

\subsection{basicEventScheduler Reference Tests}
\begin{itemize}
	\item SR2ML/tests/test\_basicEventScheduler.xml
\end{itemize}


