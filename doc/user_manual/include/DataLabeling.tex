\section{Data Labeling}
\label{sec:DataLabeling}

The \textbf{DataLabeling} post-processor is specifically used to label the data stored in the DataObjects. It
accepts two DataObjects, one DataObject (i.e., reference DataObject) with type \textbf{PointSet} is used to label the
other target DataObjects (type can be either \textbf{PointSet} or \textbf{HistorySet}).

%
\ppType{DataLabeling}{DataLabeling}
%
\begin{itemize}
  \item \xmlNode{label}, \xmlDesc{string, required field}, the label variable name in the reference DataObject.
    This variable will be used to label the target DataObject.
  \item \xmlNode{variable}, \xmlDesc{required, xml node}. In this node, the following attribute should be specified:
    \begin{itemize}
      \item \xmlAttr{name}, \xmlDesc{required, string attribute}, the variable name, which should be exist in
        the reference DataObject.
    \end{itemize}
    and the following sub-node should also be specified:
    \begin{itemize}
      \item \xmlNode{Function}, \xmlDesc{string, required field}, this function creates the mapping from
        target DataObject to the reference DataObject.
        \begin{itemize}
          \item \xmlAttr{class}, \xmlDesc{string, required field}, the class of this function (e.g. Functions)
          \item \xmlAttr{type}, \xmlDesc{string, required field}, the type of this function (e.g. external)
        \end{itemize}
    \end{itemize}
\end{itemize}
%
In order to use this post-processor, the users need to specify two different DataObjects, i.e.
\begin{lstlisting}[style=XML]
  <DataObjects>
    <PointSet name="ET_PS">
      <Input>ACC, LPI</Input>
      <Output>sequence</Output>
    </PointSet>
    <PointSet name="sim_PS">
      <Input>ACC_status, LPI_status</Input>
      <Output>out</Output>
    </PointSet>
  </DataObjects>
\end{lstlisting}
The first data object ``ET\_PS'' contains the event tree with input variables ``ACC, LPI'' and output label ``sequence''.
This data object will be used to classify the data in the second data object ``sim\_PS''. The results will be stored in
the output data object with the same label ``sequence''. Since these two data objects contain different inputs,
\xmlNode{Functions} will be used to create the maps between the inputs:
\begin{lstlisting}[style=XML]
  <Functions>
    <External file="func_ACC.py" name="func_ACC">
      <variable>ACC_status</variable>
    </External>
    <External file="func_LPI.py" name="func_LPI">
      <variable>LPI_status</variable>
    </External>
  </Functions>
\end{lstlisting}

The inputs to these functions are the data from the target DataObject, and the outputs of these functions
are the data from the reference DataObject.

\textbf{Example Python Function for ``func\_ACC.py''}
\begin{lstlisting}[language=python]
def evaluate(self):
  return self.ACC_status
\end{lstlisting}

\textbf{Example Python Function for ``func\_LPI.py''}
\begin{lstlisting}[language=python]
def evaluate(self):
  return self.LPI_status
\end{lstlisting}

\nb All the functions that are used to create the maps should be include the ``evaluate'' method.

The example of \textbf{DataLabeling} post processor is provided below:
\begin{lstlisting}[style=XML]
    <PostProcessor name="ET_Classifier" subType="SR2ML.DataLabeling">
      <label>sequence</label>
      <variable name='ACC'>
        <Function class="Functions" type="External">func_ACC</Function>
      </variable>
      <variable name='LPI'>
        <Function class="Functions" type="External">func_LPI</Function>
      </variable>
    </PostProcessor>
\end{lstlisting}
The definitions for the XML nodes can be found in the RAVEN user manual. The label ``sequence''
and the variables ``ACC, LPI'' should be exist in the reference data object,
while the functions ``func\_ACC, func\_LPI'' are used to map relationships between the reference and target data objects.

The labeling process can be achieved via the \xmlNode{Steps} as shown below:
\begin{lstlisting}[style=XML]
<Simulation>
 ...
  <Steps>
    <PostProcess name="classify">
      <Input   class="DataObjects"  type="PointSet"        >ET_PS</Input>
      <Input   class="DataObjects"  type="PointSet"        >sim_PS</Input>
      <Model   class="Models"       type="PostProcessor"   >ET_Classifier</Model>
      <Output  class="DataObjects"  type="PointSet"        >sim_PS</Output>
    </PostProcess>
  </Steps>
 ...
</Simulation>
\end{lstlisting}

\subsection{Data Labeling Reference Tests}
\begin{itemize}
	\item test\_DataLabeling\_postprocessor.xml
  \item test\_DataLabeling\_postprocessor\_HS.xml.
\end{itemize}
