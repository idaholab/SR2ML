\section{Data Labeling}
\label{sec:DataLabeling}

The \textbf{DataLabeling} post-processor is specifically used to classify the data stored in the DataObjects. It
accepts two DataObjects, one is used as the classifier which must be a \textbf{PointSet}, the other one,
i.e. \textbf{PointSet} or \textbf{HistorySet}, is used as the input DataObject to be classified.
%
\ppType{DataLabeling}{DataLabeling}
%
\begin{itemize}
  \item \xmlNode{label}, \xmlDesc{string, required field}, the name of the label that are used for the classifier. This
    label must exist in the DataObject that is used as the classifer. This name will also be used as
    the label name for the DataObject that is classified.
  \item \xmlNode{variable}, \xmlDesc{required, xml node}. In this node, the following attribute should be specified:
    \begin{itemize}
      \item \xmlAttr{name}, \xmlDesc{required, string attribute}, the variable name, which should be exist in
        the DataObject that is used as classifier.
    \end{itemize}
    and the following sub-node should also be specified:
    \begin{itemize}
      \item \xmlNode{Function}, \xmlDesc{string, required field}, this function creates the mapping from input DataObject
        to the Classifier.
        \begin{itemize}
          \item \xmlAttr{class}, \xmlDesc{string, required field}, the class of this function (e.g. Functions)
          \item \xmlAttr{type}, \xmlDesc{string, required field}, the type of this function (e.g. external)
        \end{itemize}
    \end{itemize}
\end{itemize}
%
In order to use this post-processor, the users need to specify two different DataObjects, i.e.
\begin{lstlisting}[style=XML]
  <DataObjects>
    <PointSet name="ET_PS">
      <Input>ACC, LPI</Input>
      <Output>sequence</Output>
    </PointSet>
    <PointSet name="sim_PS">
      <Input>ACC_status, LPI_status</Input>
      <Output>out</Output>
    </PointSet>
  </DataObjects>
\end{lstlisting}
The first data object ``ET\_PS'' contains the event tree with input variables ``ACC, LPI'' and output label ``sequence''.
This data object will be used to classify the data in the second data object ``sim\_PS''. The results will be stored in
the output data object with the same label ``sequence''. Since these two data objects contain different inputs,
\xmlNode{Functions} will be used to create the maps between the inputs:
\begin{lstlisting}[style=XML]
  <Functions>
    <External file="func_ACC.py" name="func_ACC">
      <variable>ACC_status</variable>
    </External>
    <External file="func_LPI.py" name="func_LPI">
      <variable>LPI_status</variable>
    </External>
  </Functions>
\end{lstlisting}

The inputs to these functions are the inputs of the data object that will be classified, and the outputs of these functions
are the inputs of data object that is used as the classifier.

\textbf{Example Python Function for ``func\_ACC.py''}
\begin{lstlisting}[language=python]
def evaluate(self):
  return self.ACC_status
\end{lstlisting}

\textbf{Example Python Function for ``func\_LPI.py''}
\begin{lstlisting}[language=python]
def evaluate(self):
  return self.LPI_status
\end{lstlisting}

\nb All the functions that are used to create the maps should be include the ``evaluate'' method.

The \textbf{DataLabeling} post-processor is specifically used to classify the data stored in the DataObjects. It
accepts two DataObjects, one is used as the classifier which should be always \textbf{PointSet}, the other one, i.e.
either \textbf{PointSet} or \textbf{HistorySet} is used as the input DataObject to be classified.

The \textbf{DataLabeling} is provided below:
\begin{lstlisting}[style=XML]
    <PostProcessor name="ET_Classifier" subType="SR2ML.DataLabeling">
      <label>sequence</label>
      <variable name='ACC'>
        <Function class="Functions" type="External">func_ACC</Function>
      </variable>
      <variable name='LPI'>
        <Function class="Functions" type="External">func_LPI</Function>
      </variable>
    </PostProcessor>
\end{lstlisting}
The definitions for the XML nodes can be found in the RAVEN user manual. The label ``sequence''
and the variables ``ACC, LPI'' should be exist in the data object that is used as the classifier,
while the functions ``func\_ACC, func\_LPI'' are used to map relationships between the input data objects.

The classification can be achieved via the \xmlNode{Steps} as shown below:
\begin{lstlisting}[style=XML]
<Simulation>
 ...
  <Steps>
    <PostProcess name="classify">
      <Input   class="DataObjects"  type="PointSet"        >ET_PS</Input>
      <Input   class="DataObjects"  type="PointSet"        >sim_PS</Input>
      <Model   class="Models"       type="PostProcessor"   >ET_Classifier</Model>
      <Output  class="DataObjects"  type="PointSet"        >sim_PS</Output>
    </PostProcess>
  </Steps>
 ...
</Simulation>
\end{lstlisting}

\subsection{Data Classifier Reference Tests}
\begin{itemize}
	\item test\_DataLabeling\_postprocessor.xml
  \item test\_DataLabeling\_postprocessor\_HS.xml.
\end{itemize}
