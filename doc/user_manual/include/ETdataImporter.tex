\section{Event Tree Data Importer}
\label{sec:ETdataImporter}

The \textbf{ETImporter} post-processor has been designed to import Event-Tree (ET) object into
RAVEN. Since several ET file formats are available, as of now only the OpenPSA format
(see https://open-psa.github.io/joomla1.5/index.php.html) is supported. As an example,
the OpenPSA format ET is shown below:

\begin{lstlisting}[style=XML,morekeywords={anAttribute},caption=ET in OpenPSA format., label=lst:ETModel]
<define-event-tree name="eventTree">
    <define-functional-event name="ACC"/>
    <define-functional-event name="LPI"/>
    <define-functional-event name="LPR"/>
    <define-sequence name="1"/>
    <define-sequence name="2"/>
    <define-sequence name="3"/>
    <define-sequence name="4"/>
    <initial-state>
        <fork functional-event="ACC">
            <path state="0">
                <fork functional-event="LPI">
                    <path state="0">
                        <fork functional-event="LPR">
                            <path state="0">
                                <sequence name="1"/>
                            </path>
                            <path state="+1">
                                <sequence name="2"/>
                            </path>
                        </fork>
                    </path>
                    <path state="+1">
                        <sequence name="3"/>
                    </path>
                </fork>
            </path>
            <path state="+1">
                <sequence name="4"/>
            </path>
        </fork>
    </initial-state>
</define-event-tree>
\end{lstlisting}

This is performed by saving the structure of the ET (from file) as a \textbf{PointSet}
(only \textbf{PointSet} are allowed), since an ET is a static Boolean logic structure. Each realization in the
\textbf{PointSet} represents a unique accident sequence of the ET, and the \textbf{PointSet} is structured as follows:
\begin{itemize}
  \item Input variables of the \textbf{PointSet} are the branching conditions of the ET. The value of each input variable can be:
  \begin{itemize}
    \item  0: event did occur (typically upper branch)
    \item  1: event did not occur (typically lower branch)
    \item -1: event is not queried (no branching occured)
  \end{itemize}
  \item Output variables of the \textbf{PointSet} are the ID of each branch of the ET (i.e., positive integers greater than 0)
\end{itemize}

\nb that the 0 or 1 values are specified in the \xmlNode{path state="0"} or \xmlNode{path state="1"} nodes in the ET OpenPSA file.

Provided this definition, the ET described in Listing~\ref{lst:ETModel} will be converted to \textbf{PointSet} that is characterized
by the following variables:
\begin{itemize}
	\item Input variables: statusACC, statusLPI, statusLPR
	\item Output variable: sequence
\end{itemize}
and the corresponding \textbf{PointSet} if the \xmlNode{expand} node is set to False is shown in Table~\ref{PointSetETExpandFalse}.
If \xmlNode{expand} set to True, the corresponding \textbf{PointSet} is shown in Table~\ref{PointSetETExpandTrue}.
\begin{table}[h]
    \centering
    \caption{PointSet generated by RAVEN by employing the ET Importer Post-Processor with \xmlNode{expand}
             set to False for the ET of Listing~\ref{lst:ETModel}.}
    \label{PointSetETExpandFalse}
	\begin{tabular}{c | c | c | c}
		\hline
		ACC & LPI & LPR & sequence \\
		\hline
		0.  &  0. &  0. & 1. \\
		0.  &  0. &  1. & 2. \\
		0.  &  1. & -1. & 3. \\
		1.  & -1. & -1. & 4. \\
		\hline
	\end{tabular}
\end{table}
\begin{table}[h]
    \centering
    \caption{PointSet generated by RAVEN by employing the ET Importer Post-Processor with \xmlNode{expand}
             set to True for the ET of Listing~\ref{lst:ETModel}.}
    \label{PointSetETExpandTrue}
	\begin{tabular}{c | c | c | c}
		\hline
		ACC & LPI & LPR & sequence \\
		\hline
		0.  &  0. &  0. & 1. \\
		0.  &  0. &  1. & 2. \\
		0.  &  1. &  0. & 3. \\
		0.  &  1. &  1. & 3. \\
		1.  &  0. &  0. & 4. \\
		1.  &  0. &  1. & 4. \\
		1.  &  1. &  0. & 4. \\
		1.  &  1. &  1. & 4. \\
		\hline
	\end{tabular}
\end{table}

The ETImporter PP supports also:
\begin{itemize}
  \item links to sub-trees
	      \nb If the ET is split in two or more ETs (and thus one file for each ET), then it is only required to list
	      all files in the Step. RAVEN automatically detect links among ETs and merge all of them into a single PointSet.
  \item by-pass branches
  \item symbolic definition of outcomes: typically outcomes are defined as either 0 (upper branch) or 1 (lower branch). If instead the ET uses the
    \textbf{success/failure} labels, then they are converted into 0/1 labels
    \nb If the branching condition is not binary or \textbf{success/failure}, then the ET Importer Post-Processor just follows
	   the numerical value of the \xmlNode{state} attribute of the \xmlNode{<path>} node in the ET OpenPSA file.
  \item symbolic/numerical definition of sequences: if the ET contains a symbolic sequence then a .xml file is generated.  This file contains
        the mapping between the sequences defined in the ET and the numerical IDs created by RAVEN. The file name is the concatenation of the ET name
        and ``\_mapping''. As an example the file ``eventTree\_mapping.xml'' generated by RAVEN:
        \begin{lstlisting}[style=XML]
            <map Tree="eventTree">
              <sequence ID="0">seq_1</sequence>
              <sequence ID="1">seq_2</sequence>
              <sequence ID="2">seq_3</sequence>
              <sequence ID="3">seq_4</sequence>
            </map>
        \end{lstlisting}
        contains the mapping of four sequences defined in the ET (seq\_1,seq\_2,seq\_3,seq\_4) with the IDs generated by RAVEN (0,1,2,3).
        Note that if the sequences defined in the ET are both numerical and symbolic then they are all mapped.
  \item The ET can contain a branch that is defined as a separate block in the \xmlNode{define-branch} node and it is
        replicated in the ET; in such case RAVEN automatically replicate such branch when generating the PointSet.
\end{itemize}
The \xmlNode{collect-formula} are not considered since this node is used to connect the Boolean formulae generated by the
Fault-Trees to the branch (i.e., fork) point.

%
\ppType{ETImporter}{ETImporter}
%
\begin{itemize}
  \item \xmlNode{fileFormat}, \xmlDesc{string, required field}, specifies the format of the file that contains the ET structure (supported format: OpenPSA).
  \item  \xmlNode{expand},\xmlDesc{bool, required parameter}, expand the ET branching conditions for all branches even if they are not queried
\end{itemize}

\textbf{Example:}
\begin{lstlisting}[style=XML,morekeywords={anAttribute},caption=ET Importer input example., label=lst:ET_PP_InputExample]
  <Files>
    <Input name="eventTreeTest" type="">eventTree.xml</Input>
  </Files>

  <Models>
    ...
    <PostProcessor name="ETimporter" subType="SR2ML.ETImporter">
      <fileFormat>OpenPSA</fileFormat>
      <expand>False</expand>
    </PostProcessor>
    ...
  </Models>

  <Steps>
    ...
    <PostProcess name="import">
      <Input   class="Files"        type=""                >eventTreeTest</Input>
      <Model   class="Models"       type="PostProcessor"   >ETimporter</Model>
      <Output  class="DataObjects"  type="PointSet"        >ET_PS</Output>
    </PostProcess>
    ...
  </Steps>

  <DataObjects>
    ...
    <PointSet name="ET_PS">
      <Input>ACC,LPI,LPR</Input>
      <Output>sequence</Output>
    </PointSet>
    ...
  </DataObjects>
\end{lstlisting}

\subsection{ET Importer Reference Tests}
\begin{itemize}
	\item test\_ETimporter.xml
	\item test\_ETimporterMultipleET.xml
	\item test\_ETimporterSymbolic.xml
	\item test\_ETimporter\_expand.xml
	\item test\_ETimporter\_DefineBranch.xml
	\item test\_ETimporter\_3branches.xml
	\item test\_ETimporter\_3branches\_NewNumbering.xml
	\item test\_ETimporter\_3branches\_NewNumbering\_expanded.xml.
\end{itemize}
