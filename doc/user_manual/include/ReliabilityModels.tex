\section{Reliability Models}
\label{sec:ReliabilityModels}

\textbf{Reliability Models/Functions} are the most frequently used in life data analysis
and reliability engineering. These models/functions give the probability of an component
operating for a certain amount of time without failure. As such, the reliability models
are function of time, in that every reliability value has an associated time value. In
other words, one must specify a time value with the desired reliability value. This degree
of flexibility makes the reliability model a much better reliability specification that the
MTTF (Mean Time To Failure), which represents only one point along the entire reliability
model.

\subsection{The Probability Density and Cumulative Density Functions}
From probability and statistics, given a continuous random variable $X$, we denote:
\begin{itemize}
	\item The probability density function (pdf), as $f(x)$.
	\item The cumulative density function (cdf), as $F(x)$.
\end{itemize}
If $x$ is a continuous random variable, then the probability of $x$ takes on a value in the
interval $[a,b]$ is the area under the pdf $f(x)$ from $a$ to $b$:
\begin{equation}
  P(a\leq x\leq b) = \int_{a}^{b} f(x)dx
\end{equation}
The cumulative distribution function is a function $F(x)$ of a random variable $x$, and is
defined for a number $x_0$ by:
\begin{equation}
  F(x_0) = P(x\leq x_0) = \int_{-\infty}^{x_0} f(s)ds
\end{equation}
That is, for a given value $x_0$, $F(x_0)$ is the probability that the observed value of $x$
will be at most $x_0$. The mathematical relationship between the pdf and cdf is given by:
\begin{equation}
  F(x) = \int_{-\infty}^{x} f(s)ds
\end{equation}
Conversely:
\begin{equation}
  f(x) = - \frac{dF(x)}{dx}
\end{equation}
The functions most commonly used in reliability engineering and life data analysis, namely the
reliability function and failure rate function, can be determined directly from the pdf definition,
or $f(t)$. Different distribuion exist, such as Lognormal, exponential, Weibull etc., and each of
them has a predefined $f(t)$. These distributions were formulated by statisticians, mathematicians
and engineers to mathematically model or represent certain behavior. Some distribution tend to better
represent life data and are most commonly referred to as lifetime distribution.

\subsection{The Reliability and Failure Rate Models}
Given the mathematical representation of a distribution, we can derive all of the functions that needed
for reliability analysis, i.e. reliability models/functions. This will only depend on the value of $t$
after the value of the distribution parameters are estimated from data.
Now, let $T$ be the random variable defining the lifetime of the component with cdf $F(t)$, which is the
time the component will operate before failure. The cdf $F(t)$ of the random variable $T$ is given by:
\begin{equation}
  F(t) = \int_{-\infty}^{t} f(T)dT
\end{equation}
If $F(t)$ is a differentiable function, then the pdf $f(t)$ is given by:
\begin{equation}
  f(t) = - \frac{dF(t)}{dt}
\end{equation}
The reliability function or survive function $R(t)$ of the component is given by:
\begin{equation}
  R(t) = P(T>t) = 1 - P(T\leq t) = 1-F(t)
\end{equation}
It is the probability that the component will operate after time t, sometimes called survival probability.
The failure rate of a system during the interval $[t,t+\Delta t]$ is the probability that a failure per
unit time occurs in the interval, given that a failure has not occurred prior to t, the beginning of the
interval. The failure rate function (i.e. instantaneous failure rate, conditional failure rate) of hazard
function is defined as the limit of the failure rate as the interval approaches zero. Hence
\begin{equation}
  \lambda (t)= \lim_{\Delta t\rightarrow 0} \frac{F(t+\Delta t) - F(t)}{\Delta tR(t)}
	 = \frac{1}{R(t)} \lim_{\Delta t\rightarrow 0} \frac{F(t+\Delta t) - F(t)}{\Delta t}
	 = \frac{1}{R(t)}\frac{dF(t)}{dt} = \frac{f(t)}{R(t)}
\end{equation}
The failure rate function is the rate of change of the conditional probability of failure at time $t$.
It measures the likelihood that a component that has operated up until time $t$ fails in the next
instant of time.
Generally $\lambda (t)$ is the one tabulated because it is measured experimentally and because it tends to
vary less rapidly with time than the other parameters. When $\lambda (t)$ is given, all other three
parameters $F(t)$, $f(t)$, $R(t)$ can be computed as follows:
\begin{equation}
  R(t) = \exp(-\int_{0}^{t} \lambda (s)ds)
\end{equation}
\begin{equation}
	f(t) = \lambda (t)R(t) = \lambda (t)\exp(-\int_{0}^{t} \lambda (s)ds)
\end{equation}
\begin{equation}
 	F(t) = 1 - R(t) = 1 - \exp(-\int_{0}^{t} \lambda (s)ds)
\end{equation}
The mean time between failure (MTBF) can be obtained by finding the expected value of the random variable
$T$, time to failure. Hence
\begin{equation}
  MTBF = E(T) = \int_{0}^{\infty} tf(t)dt = \int_{0}^{\infty} R(t)dt
\end{equation}

\subsection{The Lifetime Distributions or Aging Models}


\subsection{Reliability Models Reference Tests}
\begin{itemize}
	\item SR2ML/tests/reliabilityModel/test\_bathtub.xml
  \item SR2ML/tests/reliabilityModel/test\_erlangian.xml
	\item SR2ML/tests/reliabilityModel/test\_expon.xml
  \item SR2ML/tests/reliabilityModel/test\_exponweibull.xml
	\item SR2ML/tests/reliabilityModel/test\_fatiguelife.xml
  \item SR2ML/tests/reliabilityModel/test\_gamma.xml
	\item SR2ML/tests/reliabilityModel/test\_loglinear.xml
  \item SR2ML/tests/reliabilityModeltest\_lognorm.xml
	\item SR2ML/tests/reliabilityModel/test\_normal.xml
  \item SR2ML/tests/reliabilityModel/test\_powerlaw.xml
	\item SR2ML/tests/reliabilityModeltest\_weibull.xml
	\item SR2ML/tests/reliabilityModel/test\_time\_dep\_ensemble\_reliability.xml
\end{itemize}
