%
% This is an example LaTeX file which uses the SANDreport class file.
% It shows how a SAND report should be formatted, what sections and
% elements it should contain, and how to use the SANDreport class.
% It uses the LaTeX article class, but not the strict option.
% ItINLreport uses .eps logos and files to show how pdflatex can be used
%
% Get the latest version of the class file and more at
%    http://www.cs.sandia.gov/~rolf/SANDreport
%
% This file and the SANDreport.cls file are based on information
% contained in "Guide to Preparing {SAND} Reports", Sand98-0730, edited
% by Tamara K. Locke, and the newer "Guide to Preparing SAND Reports and
% Other Communication Products", SAND2002-2068P.
% Please send corrections and suggestions for improvements to
% Rolf Riesen, Org. 9223, MS 1110, rolf@cs.sandia.gov
%

\documentclass[pdf,12pt]{INLreport}
% pslatex is really old (1994).  It attempts to merge the times and mathptm packages.
% My opinion is that it produces a really bad looking math font.  So why are we using it?
% If you just want to change the text font, you should just \usepackage{times}.
% \usepackage{pslatex}
\usepackage{times}
\usepackage[FIGBOTCAP,normal,bf,tight]{subfigure}
\usepackage{amsmath}
\usepackage{amssymb}
\usepackage{soul}
\usepackage{pifont}
\usepackage{enumerate}
\usepackage{listings}
\usepackage{fullpage}
\usepackage{xcolor}          % Using xcolor for more robust color specification
\usepackage{ifthen}          % For simple checking in newcommand blocks
\usepackage{textcomp}
\usepackage{mathtools}
\usepackage{relsize}
\usepackage{lscape}
\usepackage[toc,page]{appendix}

\graphicspath{{./figures/}}

\newtheorem{mydef}{Definition}
\newcommand{\norm}[1]{\lVert#1\rVert}
%\usepackage[table,xcdraw]{xcolor}
%\usepackage{authblk}         % For making the author list look prettier
%\renewcommand\Authsep{,~\,}

% Custom colors
\definecolor{deepblue}{rgb}{0,0,0.5}
\definecolor{deepred}{rgb}{0.6,0,0}
\definecolor{deepgreen}{rgb}{0,0.5,0}
\definecolor{forestgreen}{RGB}{34,139,34}
\definecolor{orangered}{RGB}{239,134,64}
\definecolor{darkblue}{rgb}{0.0,0.0,0.6}
\definecolor{gray}{rgb}{0.4,0.4,0.4}

\lstset {
  basicstyle=\ttfamily,
  frame=single
}


\setcounter{secnumdepth}{5}
\lstdefinestyle{XML} {
    language=XML,
    extendedchars=true,
    breaklines=true,
    breakatwhitespace=true,
%    emph={name,dim,interactive,overwrite},
    emphstyle=\color{red},
    basicstyle=\ttfamily,
%    columns=fullflexible,
    commentstyle=\color{gray}\upshape,
    morestring=[b]",
    morecomment=[s]{<?}{?>},
    morecomment=[s][\color{forestgreen}]{<!--}{-->},
    keywordstyle=\color{cyan},
    stringstyle=\ttfamily\color{black},
    tagstyle=\color{darkblue}\bf\ttfamily,
    morekeywords={name,type},
%    morekeywords={name,attribute,source,variables,version,type,release,x,z,y,xlabel,ylabel,how,text,param1,param2,color,label},
}
\lstset{language=python,upquote=true}

\usepackage{titlesec}
\newcommand{\sectionbreak}{\clearpage}
\setcounter{secnumdepth}{4}

%\titleformat{\paragraph}
%{\normalfont\normalsize\bfseries}{\theparagraph}{1em}{}
%\titlespacing*{\paragraph}
%{0pt}{3.25ex plus 1ex minus .2ex}{1.5ex plus .2ex}

%%%%%%%% Begin comands definition to input python code into document
\usepackage[utf8]{inputenc}

% Default fixed font does not support bold face
\DeclareFixedFont{\ttb}{T1}{txtt}{bx}{n}{9} % for bold
\DeclareFixedFont{\ttm}{T1}{txtt}{m}{n}{9}  % for normal

\usepackage{listings}

% Python style for highlighting
\newcommand\pythonstyle{\lstset{
language=Python,
basicstyle=\ttm,
otherkeywords={self, none, return},             % Add keywords here
keywordstyle=\ttb\color{deepblue},
emph={MyClass,__init__},          % Custom highlighting
emphstyle=\ttb\color{deepred},    % Custom highlighting style
stringstyle=\color{deepgreen},
frame=tb,                         % Any extra options here
showstringspaces=false            %
}}


% Python environment
\lstnewenvironment{python}[1][]
{
\pythonstyle
\lstset{#1}
}
{}

% Python for external files
\newcommand\pythonexternal[2][]{{
\pythonstyle
\lstinputlisting[#1]{#2}}}

\lstnewenvironment{xml}
{}
{}

% Python for inline
\newcommand\pythoninline[1]{{\pythonstyle\lstinline!#1!}}


\def\DRAFT{} % Uncomment this if you want to see the notes people have been adding
% Comment command for developers (Should only be used under active development)
\ifdefined\DRAFT
  \newcommand{\nameLabeler}[3]{\textcolor{#2}{[[#1: #3]]}}
\else
  \newcommand{\nameLabeler}[3]{}
\fi
% Commands for making the LaTeX a bit more uniform and cleaner
\newcommand{\TODO}[1]    {\textcolor{red}{\textit{(#1)}}}
\newcommand{\xmlAttrRequired}[1] {\textcolor{red}{\textbf{\texttt{#1}}}}
\newcommand{\xmlAttr}[1] {\textcolor{cyan}{\textbf{\texttt{#1}}}}
\newcommand{\xmlNodeRequired}[1] {\textcolor{deepblue}{\textbf{\texttt{<#1>}}}}
\newcommand{\xmlNode}[1] {\textcolor{darkblue}{\textbf{\texttt{<#1>}}}}
\newcommand{\xmlString}[1] {\textcolor{black}{\textbf{\texttt{'#1'}}}}
\newcommand{\xmlDesc}[1] {\textbf{\textit{#1}}} % Maybe a misnomer, but I am
                                                % using this to detail the data
                                                % type and necessity of an XML
                                                % node or attribute,
                                                % xmlDesc = XML description
\newcommand{\default}[1]{~\\*\textit{Default: #1}}
\newcommand{\nb} {\textcolor{deepgreen}{\textbf{~Note:}}~}
\newcommand{\ppType}[2]
{
  In order to use the \textit{#1} PP, the user needs to set the
  \xmlAttr{subType} of a \xmlNode{PostProcessor} node:

  \xmlNode{PostProcessor \xmlAttr{name}=\xmlString{ppName} \xmlAttr{subType}=\xmlString{SR2ML.#2}/}.

   Several sub-nodes are available:
}

%%%%%%%% End comands definition to input python code into document

%\usepackage[dvips,light,first,bottomafter]{draftcopy}
%\draftcopyName{Sample, contains no OUO}{70}
%\draftcopyName{Draft}{300}

% The bm package provides \bm for bold math fonts.  Apparently
% \boldsymbol, which I used to always use, is now considered
% obsolete.  Also, \boldsymbol doesn't even seem to work with
% the fonts used in this particular document...
\usepackage{bm}


% Define tensors to be in bold math font.
\newcommand{\tensor}[1]{{\bm{#1}}}

% Override the formatting used by \vec.  Instead of a little arrow
% over the letter, this creates a bold character.
\renewcommand{\vec}{\bm}

% Define unit vector notation.  If you don't override the
% behavior of \vec, you probably want to use the second one.
\newcommand{\unit}[1]{\hat{\bm{#1}}}
% \newcommand{\unit}[1]{\hat{#1}}

% Use this to refer to a single component of a unit vector.
\newcommand{\scalarunit}[1]{\hat{#1}}

% \toprule, \midrule, \bottomrule for tables
\usepackage{booktabs}

% \llbracket, \rrbracket
\usepackage{stmaryrd}

\usepackage{hyperref}
\hypersetup{
    colorlinks,
    citecolor=black,
    filecolor=black,
    linkcolor=black,
    urlcolor=black
}

% Compress lists of citations like [33,34,35,36,37] to [33-37]
\usepackage{cite}

% If you want to relax some of the SAND98-0730 requirements, use the "relax"
% option. It adds spaces and boldface in the table of contents, and does not
% force the page layout sizes.
% e.g. \documentclass[relax,12pt]{SANDreport}
%
% You can also use the "strict" option, which applies even more of the
% SAND98-0730 guidelines. It gets rid of section numbers which are often
% useful; e.g. \documentclass[strict]{SANDreport}

% The INLreport class uses \flushbottom formatting by default (since
% it's intended to be two-sided document).  \flushbottom causes
% additional space to be inserted both before and after paragraphs so
% that no matter how much text is actually available, it fills up the
% page from top to bottom.  My feeling is that \raggedbottom looks much
% better, primarily because most people will view the report
% electronically and not in a two-sided printed format where some argue
% \raggedbottom looks worse.  If we really want to have the original
% behavior, we can comment out this line...
\raggedbottom
\setcounter{secnumdepth}{5} % show 5 levels of subsection
\setcounter{tocdepth}{5} % include 5 levels of subsection in table of contents

% ---------------------------------------------------------------------------- %
%
% Set the title, author, and date
%
\title{SR2ML User Manual}
%\author{%
%\begin{tabular}{c} Author 1 \\ University1 \\ Mail1 \\ \\
%Author 3 \\ University3 \\ Mail3 \end{tabular} \and
%\begin{tabular}{c} Author 2 \\ University2 \\ Mail2 \\ \\
%Author 4 \\ University4 \\ Mail4\\
%\end{tabular} }


\author{
\\Diego Mandelli
\\Congjian Wang
}

% There is a "Printed" date on the title page of a SAND report, so
% the generic \date should [WorkingDir:]generally be empty.
\date{}


% ---------------------------------------------------------------------------- %
% Set some things we need for SAND reports. These are mandatory
%
\SANDnum{INL/EXT-20-57857}
\SANDprintDate{\today}
\SANDauthor{Congjian Wang and Diego Mandelli}
\SANDreleaseType{Revision 1}
\def\component#1{\texttt{#1}}

% ---------------------------------------------------------------------------- %
\newcommand{\systemtau}{\tensor{\tau}_{\!\text{SUPG}}}

\usepackage{placeins}
\usepackage{array}

\newcolumntype{L}[1]{>{\raggedright\let\newline\\\arraybackslash\hspace{0pt}}m{#1}}
\newcolumntype{C}[1]{>{\centering\let\newline\\\arraybackslash\hspace{0pt}}m{#1}}
\newcolumntype{R}[1]{>{\raggedleft\let\newline\\\arraybackslash\hspace{0pt}}m{#1}}

% ---------------------------------------------------------------------------- %
%
% Start the document
%
\begin{document}
    \sloppy
    \maketitle

    % ------------------------------------------------------------------------ %
    % The table of contents and list of figures and tables
    % Comment out \listoffigures and \listoftables if there are no
    % figures or tables. Make sure this starts on an odd numbered page
    %
    \cleardoublepage		% TOC needs to start on an odd page
    \tableofcontents
    %\listoffigures
    %\listoftables
    % ---------------------------------------------------------------------- %
    \SANDmain

    % ---------------------------------------------------------------------- %
    % This is where the body of the report begins; usually with an Introduction
    %
    \section{Introduction}
\label{sec:Introduction}

SR2ML is a software package that contains a set of safety and reliability models
designed to be interfaced with the INL's RAVEN code. These models can be
employed to perform both static and dynamic system risk analyses and to determine
the risk importance of specific elements of the considered system.

Two classes of reliability models have been developed; the first class includes
all classical reliability models (fault trees, event trees, Markov models and
reliability block diagrams) which have been extended to not only deal with
Boolean logic values but also time-dependent values. The second class includes
several component aging and maintenance models. Models of these two classes are designed to
be included in a RAVEN ensemble model to perform a time-dependent system reliability
analysis (dynamic analysis). Similarly, these models can be interfaced with system
analysis codes  to determine the failure time of systems and evaluate the accident progression
(static analysis).

\subsection{Acquiring and Installing SR2ML}
SR2ML is supported on three separate computing platforms: Linux, OSX (Apple Macintosh), and Microsoft
Windows. Currently, SR2ML is downloadable from the SR2ML GitLab repository:
\url{https://hpcgitlab.hpc.inl.gov/RAVEN_PLUGINS/SR2ML.git}. New users should contact SR2ML developers to
get started. This typically involves the following steps:

\begin{itemize}
  \item \textit{Download SR2ML}
    \\ You can download the source code from \url{https://github.com/idaholab/SR2ML.git}.
	\item \textit{Use as a RAVEN Plugin},
    \\RAVEN must first be downloaded from \url{https://github.com/idaholab/raven.git}.
		\\ Detailed instructions are available from \url{https://github.com/idaholab/raven/wiki}.
    To register a plugin with RAVEN and make its components accessible, run the script:
\begin{lstlisting}[language=bash]
raven/scripts/install_plugins.py -s /abs/path/to/SR2ML
\end{lstlisting}
    After the plugin registration, follow the installation instructions at
    \url{https://github.com/idaholab/raven/wiki/installationMain} to install the
    required dependencies.
\end{itemize}

\subsection{Accessing SR2ML from within RAVEN}
The SR2ML can be accessed as a special subtype of the External model.
The syntax is given in Listing \ref{lst:SR2MLfromRAVEN}.

\begin{lstlisting}[style=XML,morekeywords={anAttribute},caption=Call SR2ML.FTModel from RAVEN input., label=lst:SR2MLfromRAVEN]
<ExternalModel name="FaultTree" subType="SR2ML.FTModel">
  <variables> Input and output variables needed by SR2ML </variables>
  ...
</ExternalModel>
\end{lstlisting}

\subsection{User Manual Formats}
In this manual, we employ the following formats to highlight certain parts with
particular meanings (i.e., input structure, examples, and terminal commands):

\begin{itemize}
\item \textbf{\textit{Python Coding:}}
\begin{lstlisting}[language=python]
class AClass():
  def aMethodImplementation(self):
    pass
\end{lstlisting}
\item \textbf{\textit{SR2ML XML input example:}}
\begin{lstlisting}[style=XML,morekeywords={anAttribute}]
<MainXMLBlock>
  ...
  <aXMLnode anAttribute='aValue'>
     <aSubNode>body</aSubNode>
  </aXMLnode>
  <!-- This is  commented block -->
  ...
</MainXMLBlock>
\end{lstlisting}
\item \textbf{\textit{Bash Commands:}}
\begin{lstlisting}[language=bash]
cd path/to/SR2ML/
cd ../../
\end{lstlisting}
\end{itemize}

\subsection{Capabilities of SR2ML}
This document provides a detailed description of the SR2ML plugin for the RAVEN code~\cite{RAVEN,RAVENtheoryMan}.
The features included in this plugin are:
\begin{itemize}
	\item Event Tree (ET) Model (see Section~\ref{sec:ETModel})
	\item Fault Tree (FT) Model (see Section~\ref{sec:FTModel})
	\item Markov Model (see Section~\ref{sec:MarkovModel})
	\item Reliability Block Diagram (RBD) Model (see Section~\ref{sec:RBDmodel})
	\item MCSSolver (see Section~\ref{sec:MCSSolver})
	\item Reliability Models (see Section~\ref{sec:ReliabilityModels})
  \item Maintenance Models (see Section~\ref{sec:MaintenanceModels})
	\item Data Labeling (see Section~\ref{sec:DataLabeling})
	\item ET Data Importer (see Section~\ref{sec:ETdataImporter})
	\item FT Data Importer (see Section~\ref{sec:FTdataImporter})
\end{itemize}

    %%%%%%%%%%%%% External Models %%%%%%%%%%%%%%
    \input{include/ETmodel.tex}
    \section{Fault Tree Model}
\label{sec:FTModel}

This model is designed to read the structure of the fault tree (FT) from the file and to import such Boolean logic structures as a RAVEN model.
The FT must be specified in the OpenPSA format (\href{<url>}{https://github.com/open-psa}).
As an example, the FT of Figure~\ref{fig:FT} is translated in the OpenPSA as shown below:

\begin{lstlisting}[style=XML,morekeywords={anAttribute},caption=FT in OpenPSA format., label=lst:FTModel]
<opsa-mef>
    <define-fault-tree name="FT">
        <define-gate name="TOP">
            <or>
                <gate name="G1"/>
                <gate name="G2"/>
                <gate name="G3"/>
            </or>
        </define-gate>
        <define-component name="A">
            <define-gate name="G1">
                <and>
                    <basic-event name="BE1"/>
                    <basic-event name="BE2"/>
                </and>
            </define-gate>
            <define-gate name="G2">
                <and>
                    <basic-event name="BE1"/>
                    <basic-event name="BE3"/>
                </and>
            </define-gate>
            <define-basic-event name="BE1">
                <float value="1.2e-3"/>
            </define-basic-event>
            <define-component name="B">
                <define-basic-event name="BE2">
                    <float value="2.4e-3"/>
                </define-basic-event>
                <define-basic-event name="BE3">
                    <float value="5.2e-3"/>
                </define-basic-event>
            </define-component>
        </define-component>
        <define-component name="C">
            <define-gate name="G3">
                <and>
                    <basic-event name="BE1"/>
                    <basic-event name="BE4"/>
                </and>
            </define-gate>
            <define-basic-event name="BE4">
                <float value="1.6e-3"/>
            </define-basic-event>
        </define-component>
    </define-fault-tree>
</opsa-mef>
\end{lstlisting}

\begin{lstlisting}[style=XML,morekeywords={anAttribute},caption=FT model input example., label=lst:FT_InputExample]
  <Models>
    ...
    <ExternalModel name="FT" subType="FTModel">
      <variables>
        statusBE1,statusBE2,statusBE3,statusBE4,TOP
      </variables>
      <topEvents>TOP</topEvents>
      <map var="statusBE1">BE1</map>
      <map var="statusBE2">BE2</map>
      <map var="statusBE3">BE3</map>
      <map var="statusBE4">BE4</map>
    </ExternalModel>
    ...
  </Models>
\end{lstlisting}

\begin{figure}
    \centering
    \centerline{\includegraphics[scale=0.5]{FT.pdf}}
    \caption{Example of FT.}
    \label{fig:FT}
\end{figure}

The FT of Figure~\ref{fig:FT} described in Listing~\ref{lst:FTModel} can be defined in the RAVEN input file as
shown in Listing~\ref{lst:FT_InputExample}

All the specifications of the FT model are given in the
\xmlNode{ExternalModel} block.
Inside the \xmlNode{ExternalModel} block, the XML
nodes that belong to this models are:
\begin{itemize}
  \item  \xmlNode{variables}, \xmlDesc{string, required parameter}, a list containing the names of both the input and output variables of the model
  \item  \xmlNode{topEvents}, \xmlDesc{string, required parameter}, the name of the alias Top Event
  \item  \xmlNode{map}, \xmlDesc{string, required parameter}, the name ID of the FT basic events
	  \begin{itemize}
	    \item \xmlAttr{var}, \xmlDesc{required string attribute}, the ALIAS name ID of the FT basic events
	  \end{itemize}
\end{itemize}

Provided this definition and the FT model of Figure~\ref{fig:FT} described in Listing~\ref{lst:FT_InputExample},
the resulting model in RAVEN is characterized by these variables:
\begin{itemize}
	\item Input variables: statusBE1, statusBE2, statusBE3, statusBE4
	\item Output variable: TOP
\end{itemize}

\subsection{FT Model Reference Tests}
\begin{itemize}
	\item SR2ML/tests/test\_FTmodel.xml
	\item SR2ML/tests/test\_FTmodel\_TD.xml
\end{itemize}

    \input{include/MarkovModel.tex}
    \section{Reliability Block Diagram Model}
\label{sec:RBDmodel}

This model is designed to read the structure of the reliability block diagram (RBD) from
the file and import such Boolean logic structures as a RAVEN model.
The RBD must be specified in a specific format.
As an example, the RBD of Figure~\ref{fig:RBD} is translated in the RAVEN format as shown below:

\begin{lstlisting}[style=XML,morekeywords={anAttribute},caption=RBD input file., label=lst:RBDmodel]
<Graph name="testGraph">
  <node name="CST">
    <childs>1</childs>
  </node>
  <node name="1">
    <childs>2,3,4</childs>
  </node>
  <node name="2">
    <childs>5</childs>
  </node>
  <node name="3">
    <childs>5</childs>
  </node>
  <node name="4">
    <childs>5</childs>
  </node>
  <node name="5">
    <childs>6,7,8</childs>
  </node>
  <node name="6">
    <childs>SG1</childs>
  </node>
  <node name="7">
    <childs>SG2</childs>
  </node>
  <node name="8">
    <childs>SG3</childs>
  </node>
  <node name="SG1"/>
  <node name="SG2"/>
  <node name="SG3"/>
</Graph>
\end{lstlisting}

\begin{figure}
    \centering
    \centerline{\includegraphics[scale=0.5]{RBD.pdf}}
    \caption{Example of RBD.}
    \label{fig:RBD}
\end{figure}

The FT of RBD illustrated in Figure~\ref{fig:RBD} and defined in Listing~\ref{lst:RBDmodel}
can be defined in the RAVEN input file as follows:

\begin{lstlisting}[style=XML,morekeywords={anAttribute},caption=RBD model input example., label=lst:RBD_InputExample]
  <Models>
    ...
    <ExternalModel name="graph" subType="GraphModel">
      <variables>
        status2,status3,status4,status5,
        statusSG1,statusSG2,statusSG3
      </variables>
      <modelFile>graphTest</modelFile>
      <nodesIN>CST</nodesIN>
      <nodesOUT>SG1,SG2,SG3</nodesOUT>
      <map var="status2">2</map>
      <map var="status3">3</map>
      <map var="status4">4</map>
      <map var="status5">5</map>
      <map var="statusSG1">SG1</map>
      <map var="statusSG2">SG2</map>
      <map var="statusSG3">SG3</map>
    </ExternalModel>
    ...
  </Models>
\end{lstlisting}

All the specifications of the RBD model are given in the
\xmlNode{ExternalModel} block.
Inside the \xmlNode{ExternalModel} block, the XML
nodes that belong to this model are:
\begin{itemize}
  \item  \xmlNode{variables}, \xmlDesc{string, required parameter}, a list containing the names of both the input and output variables of the model
  \item  \xmlNode{modelFile}, \xmlDesc{string, required parameter}, the name of the file that provide the RBD structure
  \item  \xmlNode{nodesIN}, \xmlDesc{string, required parameter}, the name of the input nodes
  \item  \xmlNode{nodesOUT}, \xmlDesc{string, required parameter}, the name of the output nodes
  \item  \xmlNode{map}, \xmlDesc{string, required parameter}, the name ID of the RBD node
	  \begin{itemize}
	    \item \xmlAttr{var}, \xmlDesc{required string attribute}, the ALIAS name ID of the RBD node
	  \end{itemize}
\end{itemize}

Provided this definition, the RBD model of Figure~\ref{fig:RBD} described in Listing~\ref{lst:RBDmodel},
is the resulting model in RAVEN characterized by these variables:
\begin{itemize}
	\item Input variables: status2, status3, status4, status5
	\item Output variable: statusSG1, statusSG2, statusSG3.
\end{itemize}

\subsection{RBD Model Reference Tests}
\begin{itemize}
	\item SR2ML/tests/test\_graphModel.xml
	\item SR2ML/tests/test\_graphModel\_TD.xml.
\end{itemize}

    \section{MCSSolver}
\label{sec:MCSSolver}

This model is designed to read from file a list of Minimal Cut Sets (MCSs) and to import such Boolean logic structure as a RAVEN model.
Provided the sampled Basic Events (BEs) margin or probability values, the MCSSolver determines the margin or the probability
of Top Event (TE).
The list of MCS must be provided through a CSV file with the following format:

\begin{table}
  \begin{center}
    \caption{MCS file format.}
    \label{tab:table1}
    \begin{tabular}{c|c|c}
      \textbf{ID} & \textbf{Prob} & \textbf{MCS}\\
      \hline
      1, & 0.01, & BE1\\
      2, & 0.02, & BE3\\
      3, & 0.03, & BE2,BE4\\
    \end{tabular}
  \end{center}
\end{table}

In this example:
\begin{itemize}
  \item three MCSs are defined: MCS1 = BE1, MCS2 = BE3 and MCS3 = BE2 and BE4
  \item four BEs are defined: BE1, BE2, BE3 and BE4
\end{itemize}

Note that the MCSSolver considers only the list of MCSs and it discards the rest of data contained in the csv file.

All the specifications of the MCSSolver model are given in the \xmlNode{ExternalModel} block.
Inside the \xmlNode{ExternalModel} block, the XML nodes that belong to this models are:
\begin{itemize}
  \item  \xmlNode{variables}, \xmlDesc{string, required parameter}, a list containing the names of both the input and output variables of the model
  \item  \xmlNode{solver}
	  \begin{itemize}
	    \item \xmlAttr{type}, \xmlDesc{required string attribute}, type of calculation to be performed: probability or margin

        \item \xmlNode{solverOrder},\xmlDesc{integer, required parameter}, solver order of $P(TE)$ for probability calculations: it specifies the
                                                                           maximum calculation envelope for $P(TE)$, i.e., the maximum number of MCSs
                                                                           to be considered when evaluating the probability of their union
         \item \xmlNode{metric},\xmlDesc{string, required parameter}, metric used for margin calculations of the top event of $M(TE)$; it specifies the
                                                                      desired distance metric in terms of p values for the $L_p$ norm (allowed values
                                                                      are: 0, 1, 2, inf). 
          \item \xmlNode{setType},\xmlDesc{string, required parameter}, type of provided sets : minimal path sets (path) or minimal cut sets (cut)                                                            
      \end{itemize}
  \item  \xmlNode{topEventID},\xmlDesc{string, required parameter}, the name of the alias variable for the Top Event
  \item  \xmlNode{map},\xmlDesc{string, required parameter}, the name ID of the ET branching variable
	  \begin{itemize}
	    \item \xmlAttr{var}, \xmlDesc{required string attribute}, the ALIAS name ID of the basic event
	  \end{itemize}
  \item \xmlNode{fileFrom}, \xmlDesc{string, optional parameter}, either `None' or `saphire', indicates where the MCSs file is coming from. Currently,
  we only support normal csv file and file generated by Saphire code. Default is `None', which means the user need to provide
  the normal csv file as mentioned above.
\end{itemize}

An example of RAVEN input file is the following:

\begin{lstlisting}[style=XML,morekeywords={anAttribute},caption=MCSSolver model input example., label=lst:MCSSolver_InputExample]
  <Models>
    ...
    <Models>
      <ExternalModel name="MCSmodel" subType="SR2ML.MCSSolver">
        <variables>
          statusBE1,statusBE2,statusBE3,statusBE4,TOP
        </variables>
        <solver type='probability'>
          <solverOrder>3</solverOrder>
        </solver>
        <topEventID>TOP</topEventID>>
        <map var='pBE1'>BE1</map>
        <map var='pBE2'>BE2</map>
        <map var='pBE3'>BE3</map>
        <map var='pBE4'>BE4</map>
      </ExternalModel>
    </Models>
    ...
  </Models>
\end{lstlisting}

In this case, the MCSs are written in terms of the variables BE1, BE2, BE3, and BE4.
The values of the variables pBE1, pBE2, pBE3, and pBE4 (i.e., the probability values associated to each basic event) are
generated outside the MCSSolver model (e.g., by a sampler) and passed to the MCSSolver model in order to calculate the
probability value of the top event (e.g., the variable TOP in the example above).
The \xmlNode{map} blocks allow the user to link the sampled probability value to the correspoding basic event.

If $solverOrder=1$ then: $P(TE) = P(MCS1)+P(MCS2)+P(MCS3)$.
If $solverOrder=2$ then: $P(TE) = P(MCS1)+P(MCS2)+P(MCS3) - P(MCS1 MCS2) - P(MCS1 MCS3) - P(MCS2 MCS3)$.
If $solverOrder=3$ then: $P(TE) = P(MCS1)+P(MCS2)+P(MCS3) - P(MCS1 MCS2) - P(MCS1 MCS3) - P(MCS2 MCS3) + P(MCS1 MCS2 MCS3)$

\subsection{Time dependent calculation}

The MCSSolver can also perform time dependent calculation by providing in the MultiRun step a dataObject which
contains the logic status of the Basic Events.
This dataObject contains the logical status of the the Basic Event: 0 (Basic event set to False: probability=p(BE))
or 1 (Basic event set to True: probability=1.0).
The format of the dataObject can be:
\begin{itemize}
  \item HistorySet: it contains the temporal profile of each basic event (e.g., BE1, BE2, BE3, and BE4) as a time
                    series (see test\_MCSSolver\_TD.xml). In this case it is needed to specify the ID of the time variable
                    contained in the HistorySet in the \xmlNode{timeID} node.

   \begin{lstlisting}[style=XML,morekeywords={anAttribute},caption=Time dependent (from HistorySet) MCSSolver model input example., label=lst:MCSSolver_InputExample]
     <Models>
      <ExternalModel name="MCSmodel" subType="SR2ML.MCSSolver">
        <variables>
          statusBE1,statusBE2,statusBE3,statusBE4,TOP,time
        </variables>
        <solverOrder>3</solverOrder>
        <topEventID>TOP</topEventID>
        <timeID>time</timeID>
        <map var='statusBE1'>BE1</map>
        <map var='statusBE2'>BE2</map>
        <map var='statusBE3'>BE3</map>
        <map var='statusBE4'>BE4</map>
      </ExternalModel>
    </Models>
  \end{lstlisting}

  \item PointSet: it contains the interval time (initial and final time) under which each basic event is set to
                  True (see test\_MCSSolver\_TD\_fromPS.xml). As an example, the PointSet contains the IDs of the basic events
                  in the and the initial and final time as shown below:

   \begin{lstlisting}[style=XML,morekeywords={anAttribute},caption= Example of PointSet for time dependent MCSSolver calculation., label=lst:MCSSolver_InputExample]
    <PointSet name="maintenanceSchedule_PointSet">
      <Input>BE</Input>
      <Output>tIn,tFin</Output>
    </PointSet>
    \end{lstlisting}

                  In this case, the MCSSolver requires the specification of the DataObject variables that contain the list of basic events
                  (\xmlNode{BE\_ID} node) and the initial and final time values (\xmlNode{tInitial} and \xmlNode{tEnd} nodes).

   \begin{lstlisting}[style=XML,morekeywords={anAttribute},caption=Time dependent (from PointSet) MCSSolver model input example., label=lst:MCSSolver_InputExample]
    <Models>
      <ExternalModel name="MCSmodel" subType="SR2ML.MCSSolver">
        <variables>
          pBE1,pBE2,pBE3,pBE4,TOP,time,BE1,BE2,BE3,BE4
        </variables>
        <solverOrder>3</solverOrder>
        <BE_ID>BE</BE_ID>
        <tInitial>tIn</tInitial>
        <tEnd>tFin</tEnd>
        <topEventID>TOP</topEventID>
        <timeID>time</timeID>
        <map var='pBE1'>BE1</map>
        <map var='pBE2'>BE2</map>
        <map var='pBE3'>BE3</map>
        <map var='pBE4'>BE4</map>
      </ExternalModel>
    </Models>
  \end{lstlisting}
\end{itemize}

\subsection{MCSSolver model reference tests}
The following is the provided analytic tests:
\begin{itemize}
  \item test\_MCSSolver.xml
  \item test\_MCSSolver\_TD.xml
  \item test\_MCSSolver\_TD\_fromPS.xml
\end{itemize}

    \input{include/ReliabilityModels.tex}
    \input{include/MaintenanceModels.tex}
    %%%%%%%%%%%%% PostProcessors %%%%%%%%%%%%%%%%
    \section{Data Labeling}
\label{sec:DataLabeling}

The \textbf{DataLabeling} post-processor is specifically used to label the data stored in the DataObjects. It
accepts two DataObjects, one DataObject (i.e., reference DataObject) with type \textbf{PointSet} is used to label the
other target DataObjects (type can be either \textbf{PointSet} or \textbf{HistorySet}).

%
\ppType{DataLabeling}{DataLabeling}
%
\begin{itemize}
  \item \xmlNode{label}, \xmlDesc{string, required field}, the label variable name in the reference DataObject.
    This variable will be used to label the target DataObject.
  \item \xmlNode{variable}, \xmlDesc{required, xml node}. In this node, the following attribute should be specified:
    \begin{itemize}
      \item \xmlAttr{name}, \xmlDesc{required, string attribute}, the variable name, which should be exist in
        the reference DataObject.
    \end{itemize}
    and the following sub-node should also be specified:
    \begin{itemize}
      \item \xmlNode{Function}, \xmlDesc{string, required field}, this function creates the mapping from
        target DataObject to the reference DataObject.
        \begin{itemize}
          \item \xmlAttr{class}, \xmlDesc{string, required field}, the class of this function (e.g. Functions)
          \item \xmlAttr{type}, \xmlDesc{string, required field}, the type of this function (e.g. external)
        \end{itemize}
    \end{itemize}
\end{itemize}
%
In order to use this post-processor, the users need to specify two different DataObjects, i.e.
\begin{lstlisting}[style=XML]
  <DataObjects>
    <PointSet name="ET_PS">
      <Input>ACC, LPI</Input>
      <Output>sequence</Output>
    </PointSet>
    <PointSet name="sim_PS">
      <Input>ACC_status, LPI_status</Input>
      <Output>out</Output>
    </PointSet>
  </DataObjects>
\end{lstlisting}
The first data object ``ET\_PS'' contains the event tree with input variables ``ACC, LPI'' and output label ``sequence''.
This data object will be used to classify the data in the second data object ``sim\_PS''. The results will be stored in
the output data object with the same label ``sequence''. Since these two data objects contain different inputs,
\xmlNode{Functions} will be used to create the maps between the inputs:
\begin{lstlisting}[style=XML]
  <Functions>
    <External file="func_ACC.py" name="func_ACC">
      <variable>ACC_status</variable>
    </External>
    <External file="func_LPI.py" name="func_LPI">
      <variable>LPI_status</variable>
    </External>
  </Functions>
\end{lstlisting}

The inputs to these functions are the data from the target DataObject, and the outputs of these functions
are the data from the reference DataObject.

\textbf{Example Python Function for ``func\_ACC.py''}
\begin{lstlisting}[language=python]
def evaluate(self):
  return self.ACC_status
\end{lstlisting}

\textbf{Example Python Function for ``func\_LPI.py''}
\begin{lstlisting}[language=python]
def evaluate(self):
  return self.LPI_status
\end{lstlisting}

\nb All the functions that are used to create the maps should be include the ``evaluate'' method.

The example of \textbf{DataLabeling} post processor is provided below:
\begin{lstlisting}[style=XML]
    <PostProcessor name="ET_Classifier" subType="SR2ML.DataLabeling">
      <label>sequence</label>
      <variable name='ACC'>
        <Function class="Functions" type="External">func_ACC</Function>
      </variable>
      <variable name='LPI'>
        <Function class="Functions" type="External">func_LPI</Function>
      </variable>
    </PostProcessor>
\end{lstlisting}
The definitions for the XML nodes can be found in the RAVEN user manual. The label ``sequence''
and the variables ``ACC, LPI'' should be exist in the reference data object,
while the functions ``func\_ACC, func\_LPI'' are used to map relationships between the reference and target data objects.

The labeling process can be achieved via the \xmlNode{Steps} as shown below:
\begin{lstlisting}[style=XML]
<Simulation>
 ...
  <Steps>
    <PostProcess name="classify">
      <Input   class="DataObjects"  type="PointSet"        >ET_PS</Input>
      <Input   class="DataObjects"  type="PointSet"        >sim_PS</Input>
      <Model   class="Models"       type="PostProcessor"   >ET_Classifier</Model>
      <Output  class="DataObjects"  type="PointSet"        >sim_PS</Output>
    </PostProcess>
  </Steps>
 ...
</Simulation>
\end{lstlisting}

\subsection{Data Labeling Reference Tests}
\begin{itemize}
	\item test\_DataLabeling\_postprocessor.xml
  \item test\_DataLabeling\_postprocessor\_HS.xml.
\end{itemize}

    \section{Event Tree Data Importer}
\label{sec:ETdataImporter}

The \textbf{ETImporter} post-processor has been designed to import Event-Tree (ET) object into
RAVEN. Since several ET file formats are available, as of now only the OpenPSA format
(see https://open-psa.github.io/joomla1.5/index.php.html) is supported. As an example,
the OpenPSA format ET is shown below:

\begin{lstlisting}[style=XML,morekeywords={anAttribute},caption=ET in OpenPSA format., label=lst:ETModel]
<define-event-tree name="eventTree">
    <define-functional-event name="ACC"/>
    <define-functional-event name="LPI"/>
    <define-functional-event name="LPR"/>
    <define-sequence name="1"/>
    <define-sequence name="2"/>
    <define-sequence name="3"/>
    <define-sequence name="4"/>
    <initial-state>
        <fork functional-event="ACC">
            <path state="0">
                <fork functional-event="LPI">
                    <path state="0">
                        <fork functional-event="LPR">
                            <path state="0">
                                <sequence name="1"/>
                            </path>
                            <path state="+1">
                                <sequence name="2"/>
                            </path>
                        </fork>
                    </path>
                    <path state="+1">
                        <sequence name="3"/>
                    </path>
                </fork>
            </path>
            <path state="+1">
                <sequence name="4"/>
            </path>
        </fork>
    </initial-state>
</define-event-tree>
\end{lstlisting}

This is performed by saving the structure of the ET (from file) as a \textbf{PointSet}
(only \textbf{PointSet} are allowed), since an ET is a static Boolean logic structure. Each realization in the
\textbf{PointSet} represents a unique accident sequence of the ET, and the \textbf{PointSet} is structured as follows:
\begin{itemize}
  \item Input variables of the \textbf{PointSet} are the branching conditions of the ET. The value of each input variable can be:
  \begin{itemize}
    \item  0: event did occur (typically upper branch)
    \item  1: event did not occur (typically lower branch)
    \item -1: event is not queried (no branching occured)
  \end{itemize}
  \item Output variables of the \textbf{PointSet} are the ID of each branch of the ET (i.e., positive integers greater than 0)
\end{itemize}

\nb that the 0 or 1 values are specified in the \xmlNode{path state="0"} or \xmlNode{path state="1"} nodes in the ET OpenPSA file.

Provided this definition, the ET described in Listing~\ref{lst:ETModel} will be converted to \textbf{PointSet} that is characterized
by the following variables:
\begin{itemize}
	\item Input variables: statusACC, statusLPI, statusLPR
	\item Output variable: sequence
\end{itemize}
and the corresponding \textbf{PointSet} if the \xmlNode{expand} node is set to False is shown in Table~\ref{PointSetETExpandFalse}.
If \xmlNode{expand} set to True, the corresponding \textbf{PointSet} is shown in Table~\ref{PointSetETExpandTrue}.
\begin{table}[h]
    \centering
    \caption{PointSet generated by RAVEN by employing the ET Importer Post-Processor with \xmlNode{expand}
             set to False for the ET of Listing~\ref{lst:ETModel}.}
    \label{PointSetETExpandFalse}
	\begin{tabular}{c | c | c | c}
		\hline
		ACC & LPI & LPR & sequence \\
		\hline
		0.  &  0. &  0. & 1. \\
		0.  &  0. &  1. & 2. \\
		0.  &  1. & -1. & 3. \\
		1.  & -1. & -1. & 4. \\
		\hline
	\end{tabular}
\end{table}
\begin{table}[h]
    \centering
    \caption{PointSet generated by RAVEN by employing the ET Importer Post-Processor with \xmlNode{expand}
             set to True for the ET of Listing~\ref{lst:ETModel}.}
    \label{PointSetETExpandTrue}
	\begin{tabular}{c | c | c | c}
		\hline
		ACC & LPI & LPR & sequence \\
		\hline
		0.  &  0. &  0. & 1. \\
		0.  &  0. &  1. & 2. \\
		0.  &  1. &  0. & 3. \\
		0.  &  1. &  1. & 3. \\
		1.  &  0. &  0. & 4. \\
		1.  &  0. &  1. & 4. \\
		1.  &  1. &  0. & 4. \\
		1.  &  1. &  1. & 4. \\
		\hline
	\end{tabular}
\end{table}

The ETImporter PP supports also:
\begin{itemize}
  \item links to sub-trees
	      \nb If the ET is split in two or more ETs (and thus one file for each ET), then it is only required to list
	      all files in the Step. RAVEN automatically detect links among ETs and merge all of them into a single PointSet.
  \item by-pass branches
  \item symbolic definition of outcomes: typically outcomes are defined as either 0 (upper branch) or 1 (lower branch). If instead the ET uses the
    \textbf{success/failure} labels, then they are converted into 0/1 labels
    \nb If the branching condition is not binary or \textbf{success/failure}, then the ET Importer Post-Processor just follows
	   the numerical value of the \xmlNode{state} attribute of the \xmlNode{<path>} node in the ET OpenPSA file.
  \item symbolic/numerical definition of sequences: if the ET contains a symbolic sequence then a .xml file is generated.  This file contains
        the mapping between the sequences defined in the ET and the numerical IDs created by RAVEN. The file name is the concatenation of the ET name
        and ``\_mapping''. As an example the file ``eventTree\_mapping.xml'' generated by RAVEN:
        \begin{lstlisting}[style=XML]
            <map Tree="eventTree">
              <sequence ID="0">seq_1</sequence>
              <sequence ID="1">seq_2</sequence>
              <sequence ID="2">seq_3</sequence>
              <sequence ID="3">seq_4</sequence>
            </map>
        \end{lstlisting}
        contains the mapping of four sequences defined in the ET (seq\_1,seq\_2,seq\_3,seq\_4) with the IDs generated by RAVEN (0,1,2,3).
        Note that if the sequences defined in the ET are both numerical and symbolic then they are all mapped.
  \item The ET can contain a branch that is defined as a separate block in the \xmlNode{define-branch} node and it is
        replicated in the ET; in such case RAVEN automatically replicate such branch when generating the PointSet.
\end{itemize}
The \xmlNode{collect-formula} are not considered since this node is used to connect the Boolean formulae generated by the
Fault-Trees to the branch (i.e., fork) point.

%
\ppType{ETImporter}{ETImporter}
%
\begin{itemize}
  \item \xmlNode{fileFormat}, \xmlDesc{string, required field}, specifies the format of the file that contains the ET structure (supported format: OpenPSA).
  \item  \xmlNode{expand},\xmlDesc{bool, required parameter}, expand the ET branching conditions for all branches even if they are not queried
\end{itemize}

\textbf{Example:}
\begin{lstlisting}[style=XML,morekeywords={anAttribute},caption=ET Importer input example., label=lst:ET_PP_InputExample]
  <Files>
    <Input name="eventTreeTest" type="">eventTree.xml</Input>
  </Files>

  <Models>
    ...
    <PostProcessor name="ETimporter" subType="SR2ML.ETImporter">
      <fileFormat>OpenPSA</fileFormat>
      <expand>False</expand>
    </PostProcessor>
    ...
  </Models>

  <Steps>
    ...
    <PostProcess name="import">
      <Input   class="Files"        type=""                >eventTreeTest</Input>
      <Model   class="Models"       type="PostProcessor"   >ETimporter</Model>
      <Output  class="DataObjects"  type="PointSet"        >ET_PS</Output>
    </PostProcess>
    ...
  </Steps>

  <DataObjects>
    ...
    <PointSet name="ET_PS">
      <Input>ACC,LPI,LPR</Input>
      <Output>sequence</Output>
    </PointSet>
    ...
  </DataObjects>
\end{lstlisting}

\subsection{ET Importer Reference Tests}
\begin{itemize}
	\item test\_ETimporter.xml
	\item test\_ETimporterMultipleET.xml
	\item test\_ETimporterSymbolic.xml
	\item test\_ETimporter\_expand.xml
	\item test\_ETimporter\_DefineBranch.xml
	\item test\_ETimporter\_3branches.xml
	\item test\_ETimporter\_3branches\_NewNumbering.xml
	\item test\_ETimporter\_3branches\_NewNumbering\_expanded.xml.
\end{itemize}

    \section{Fault Tree Data Importer}
\label{sec:FTdataImporter}

The \textbf{FTImporter} post-processor has been designed to import Fault-Tree (FT) object into
RAVEN. Since several FT file formats are available, as of now only the OpenPSA format
(see https://open-psa.github.io/joomla1.5/index.php.html) is supported. As an example,
the FT in OpenPSA format is shown in Listing~\ref{lst:FTModel}.

\begin{lstlisting}[style=XML,morekeywords={anAttribute},caption=FT in OpenPSA format., label=lst:FTModel]
<opsa-mef>
    <define-fault-tree name="FT">
        <define-gate name="TOP">
            <or>
                <gate name="G1"/>
                <gate name="G2"/>
                <gate name="G3"/>
            </or>
        </define-gate>
        <define-component name="A">
            <define-gate name="G1">
                <and>
                    <basic-event name="BE1"/>
                    <basic-event name="BE2"/>
                </and>
            </define-gate>
            <define-gate name="G2">
                <and>
                    <basic-event name="BE1"/>
                    <basic-event name="BE3"/>
                </and>
            </define-gate>
            <define-basic-event name="BE1">
                <float value="1.2e-3"/>
            </define-basic-event>
            <define-component name="B">
                <define-basic-event name="BE2">
                    <float value="2.4e-3"/>
                </define-basic-event>
                <define-basic-event name="BE3">
                    <float value="5.2e-3"/>
                </define-basic-event>
            </define-component>
        </define-component>
        <define-component name="C">
            <define-gate name="G3">
                <and>
                    <basic-event name="BE1"/>
                    <basic-event name="BE4"/>
                </and>
            </define-gate>
            <define-basic-event name="BE4">
                <float value="1.6e-3"/>
            </define-basic-event>
        </define-component>
    </define-fault-tree>
</opsa-mef>
\end{lstlisting}

This is performed by saving the structure of the FT (from file) as a \textbf{PointSet}
(only \textbf{PointSet} are allowed).

Each Point in the PointSet represents a unique combination of the basic events.
The PointSet is structured as follows: input variables are the basic events, output variable is the top event of the FT.
The value for each input and output variable can have the following values:
\begin{itemize}
  \item  0: False
  \item  1: True
\end{itemize}

Provided this definition, the FT model of Listing~\ref{lst:FTModel} can be converted to
\textbf{PointSet} that is characterized by these variables:
\begin{itemize}
	\item Input variables: BE1, BE2, BE3, BE4
	\item Output variable: out
\end{itemize}
and it is structured is shown in Table~\ref{PointSetFT}.

\begin{table}[h]
    \centering
    \caption{PointSet generated by RAVEN by employing the FT Importer Post-Processor for the FT of Listing~\ref{lst:FTModel}.}
    \label{PointSetFT}
	\begin{tabular}{c | c | c | c | c}
		\hline
		BE1 & BE2 & BE3 & BE4 & TOP \\
		\hline
		 0. &  0. &  0. &  0. &  0. \\
		 0. &  0. &  0. &  1. &  0. \\
		 0. &  0. &  1. &  0. &  0. \\
		 0. &  0. &  1. &  1. &  0. \\
		 0. &  1. &  0. &  0. &  0. \\
		 0. &  1. &  0. &  1. &  0. \\
		 0. &  1. &  1. &  0. &  0. \\
		 0. &  1. &  1. &  1. &  0. \\
		 1. &  0. &  0. &  0. &  0. \\
		 1. &  0. &  0. &  1. &  1. \\
		 1. &  0. &  1. &  0. &  1. \\
		 1. &  0. &  1. &  1. &  1. \\
		 1. &  1. &  0. &  0. &  1. \\
		 1. &  1. &  0. &  1. &  1. \\
		 1. &  1. &  1. &  0. &  1. \\
		 1. &  1. &  1. &  1. &  1. \\
		\hline
	\end{tabular}
\end{table}

%
\ppType{FTImporter}{FTImporter}
%
\begin{itemize}
  \item \xmlNode{fileFormat}, \xmlDesc{string, required field}, specifies the format of the file that contains the
    FT structure (supported format: OpenPSA).
  \item  \xmlNode{topEventID},\xmlDesc{string, required parameter}, the name of the top event of the FT
\end{itemize}

The example of FTImporter PostProcessor is shown in Listing~\ref{lst:FT_PP_InputExample}
\begin{lstlisting}[style=XML,morekeywords={anAttribute},caption=FT Importer input example., label=lst:FT_PP_InputExample]
  <Files>
    <Input name="faultTreeTest" type="">FTimporter_not.xml</Input>
  </Files>

  <Models>
    ...
    <PostProcessor name="FTimporter" subType="SR2ML.FTImporter">
      <fileFormat>OpenPSA</fileFormat>
      <topEventID>TOP</topEventID>
    </PostProcessor>
    ...
  </Models>

  <Steps>
    ...
    <PostProcess name="import">
      <Input   class="Files"        type=""                >faultTreeTest</Input>
      <Model   class="Models"       type="PostProcessor"   >FTimporter</Model>
      <Output  class="DataObjects"  type="PointSet"        >FT_PS</Output>
    </PostProcess>
    ...
  </Steps>

  <DataObjects>
    ...
    <PointSet name="FT_PS">
      <Input>BE1,BE2,BE3,BE4</Input>
      <Output>TOP</Output>
    </PointSet>
    ...
  </DataObjects>
\end{lstlisting}

Important notes and capabilities:
\begin{itemize}
	\item If the FT is split in two or more FTs (and thus one file for each FT), then it is only required to list
	      all files in the Step. RAVEN automatically detect links among FTs and merge all of them into a single PointSet.
	\item Allowed gates: AND, OR, NOT, ATLEAST, CARDINALITY, IFF, imply, NAND, NOR, XOR
	\item If an house-event is defined in the FT:
\begin{lstlisting}[style=XML,morekeywords={anAttribute},caption=FT Importer input example: house-event., label=lst:FT_house event]
<opsa-mef>
    <define-fault-tree name="FT">
        <define-gate name="TOP">
            <or>
                <basic-event name="BE1"/>
                <basic-event name="BE2"/>
                <house-event name="HE1"/>
            </or>
        </define-gate>
        <define-house-event name="HE1">
        	<constant value="true"/>
        </define-house-event>
    </define-fault-tree>
</opsa-mef>
\end{lstlisting}
           then the HE1 is not part of the PointSet (value is fixed)
\end{itemize}

\subsection{FT Importer Reference Tests}
\begin{itemize}
	\item test\_FTimporter\_and\_withNOT\_embedded.xml
	\item test\_FTimporter\_and\_withNOT\_withNOT\_embedded.xml
	\item test\_FTimporter\_and\_withNOT.xml
	\item test\_FTimporter\_and.xml
	\item test\_FTimporter\_atleast.xml
	\item test\_FTimporter\_cardinality.xml
	\item test\_FTimporter\_component.xml
	\item test\_FTimporter\_doubleNot.xml
	\item test\_FTimporter\_iff.xml
	\item test\_FTimporter\_imply.xml
	\item test\_FTimporter\_multipleFTs.xml
	\item test\_FTimporter\_nand.xml
	\item test\_FTimporter\_nor.xml
	\item test\_FTimporter\_not.xml
	\item test\_FTimporter\_or\_houseEvent.xml
	\item test\_FTimporter\_or.xml
	\item test\_FTimporter\_xor.xml.
\end{itemize}

    \section{MCSImporter}
\label{MCSimporterPP}

The \textbf{MCSImporter} post-processor has been designed to import Minimal Cut Sets (MCSs) into RAVEN.
This post-processor reads a csv file which contain the list of MCSs and it save this list as a DataObject
(i.e.,  a PointSet).
The csv file is composed by three columns; the first contains the ID number of the MCS, the second one contains
the MCS probability value, the third one lists all the Basic Events contained in the MCS.
An example of csv file is shown in Table~\ref{MCScsv}.

\begin{table}[h]
    \centering
    \caption{Example of csv file which contains four MCSs.}
    \label{MCScsv}
	\begin{tabular}{c  c  c}
		\hline
		ID, & Prob, & MCS, \\
		\hline
		1.,  &  1.8E-2, &  D  \\
		2.,  &  4.0E-3, &  B \\
		3.,  &  3.0E-4, &  A,C  \\
		4.,  &  2.1E-5, &  E,C \\
		\hline
	\end{tabular}
\end{table}

The PointSet is structured to include all Basic Event, the MCS ID, the MCS probability, and the outcome of such MCS
(always set to 1).
MCS ID and MCS probability are copied directly from the csv file.
For each MCS, the Basic Events can have two possible values:
  \begin{itemize}
    \item  0: Basic Event is not included in the MCS
    \item  1: Basic Event is included in the MCS
  \end{itemize}
The PointSet generated from the csv file of Table~\ref{MCScsv} is shown in Table~\ref{PointSetMCSExpandFalse}.
\begin{table}[h]
    \centering
    \caption{PointSet generated by RAVEN for the list of MCSs shown in Table~\ref{MCScsv}.}
    \label{PointSetMCSExpandFalse}
	\begin{tabular}{c | c | c | c | c | c | c | c }
		\hline
		A & B & C & D & E & MCS\_ID & probability & out \\
		\hline
		0 & 0 & 0 & 1 & 0 & 1 & 1.8E-2 & 1 \\
		0 & 1 & 0 & 0 & 0 & 2 & 4.0E-3 & 1 \\
		1 & 0 & 1 & 0 & 0 & 3 & 3.0E-4 & 1 \\
		0 & 0 & 1 & 0 & 1 & 4 & 4.0E-3 & 1 \\
		\hline
	\end{tabular}
\end{table}

%
\ppType{MCSImporter}{MCSImporter}
%
\begin{itemize}
  \item  \xmlNode{expand},\xmlDesc{bool, required parameter}, expand the set of Basic Events by including all PRA Basic Events
  and not only the once listed in the MCSs
  \item  \xmlNode{BElistColumn},\xmlDesc{string, optional parameter}, if expand is set to True, then this node contains the
  column of the csv file which contains all the PRA Basic Events
  \item \xmlNode{fileFrom}, \xmlDesc{string, optional parameter}, either `None' or `saphire', indicates where the MCSs file is coming from. Currently,
  we only support normal csv file and file generated by Saphire code. Default is `None', which means the user need to provide
  the normal csv file as mentioned above.
\end{itemize}

\textbf{Example:}
\begin{lstlisting}[style=XML,morekeywords={anAttribute},caption=MCS Importer input example (no expand)., label=lst:MCS_PP_InputExample]
  <Files>
    <Input name="MCSlistFile" type="MCSlist">MCSlist.csv</Input>
  </Files>

  <Models>
    <PostProcessor name="MCSImporter" subType="SR2ML.MCSImporter">
      <expand>False</expand>
    </PostProcessor>
  </Models>

  <Steps>
    <PostProcess name="import">
      <Input   class="Files"        type="MCSlist"         >MCSlistFile</Input>
      <Model   class="Models"       type="PostProcessor"   >MCSImporter</Model>
      <Output  class="DataObjects"  type="PointSet"        >MCS_PS</Output>
    </PostProcess>
  </Steps>

  <DataObjects>
    <PointSet name="MCS_PS">
      <Input>A,B,C,D,E</Input>
      <Output>MCS_ID,probability,out</Output>
    </PointSet>
  </DataObjects>
\end{lstlisting}

\textbf{Example:}
\begin{lstlisting}[style=XML,morekeywords={anAttribute},caption=MCS Importer input example (expanded)., label=lst:MCS_PP_InputExample]
  <Files>
    <Input name="MCSlistFile" type="MCSlist">MCSlist.csv</Input>
    <Input name="BElistFile"  type="BElist" >BElist.csv</Input>
  </Files>

  <Models>
    <PostProcessor name="MCSImporter" subType="SR2ML.MCSImporter">
      <expand>False</expand>
    </PostProcessor>
  </Models>

  <Steps>
    <PostProcess name="import">
      <Input   class="Files"        type="MCSlist"         >MCSlistFile</Input>
      <Input   class="Files"        type="BElist"          >BElistFile</Input>
      <Model   class="Models"       type="PostProcessor"   >MCSimporter</Model>
      <Output  class="DataObjects"  type="PointSet"        >MCS_PS</Output>
    </PostProcess>
  </Steps>

  <DataObjects>
    <PointSet name="MCS_PS">
      <Input>A,B,C,D,E,F,G</Input>
      <Output>MCS_ID,probability,out</Output>
    </PointSet>
  </DataObjects>
\end{lstlisting}

    \section{Discrete Risk Measures}
\label{DiscreteRiskMeasures}
This Post-Processor calculates a series of risk importance measures from a PointSet.
This calculation is performed for a set of input parameters given an output target.

The user is required to provide the following information:
\begin{itemize}
   \item the set of input variables. For each variable the following need to be specified:
     \begin{itemize}
       \item the set of values that imply a reliability value equal to $1$ for the input variable
       \item the set of values that imply a reliability value equal to $0$ for the input variable
     \end{itemize}
   \item the output target variable. For this variable it is needed to specify the values of
      the output target variable that defines the desired outcome.
\end{itemize}

The following variables are first determined for each input variable $i$:
\begin{itemize}
   \item $R_0$ Probability of the outcome of the output target variable (nominal value)
   \item $R^{+}_i$ Probability of the outcome of the output target variable if reliability of the input variable is equal to $0$
   \item $R^{-}_i$ Probability of the outcome of the output target variable if reliability of the input variable is equal to $1$
\end{itemize}

Available measures are:
\begin{itemize}
   \item Risk Achievement Worth (RAW): $RAW = R^{+}_i / R_0 $
   \item Risk Achievement Worth (RRW): $RRW = R_0 / R^{-}_i$
   \item Fussell-Vesely (FV): $FV = (R_0 - R^{-}_i) / R_0$
   \item Birnbaum (B): $B = R^{+}_i - R^{-}_i$
\end{itemize}

\ppType{RiskMeasureDiscrete}{RiskMeasureDiscrete}

In the \xmlNode{PostProcessor} input block, the following XML sub-nodes are required,
independent of the \xmlAttr{subType} specified:

\begin{itemize}
   \item \xmlNode{measures}, \xmlDesc{string, required field}, desired risk importance measures
      that have to be computed (RRW, RAW, FV, B)
   \item \xmlNode{variable}, \xmlDesc{string, required field}, ID of the input variable. This
      node is provided for each input variable. This nodes needs to contain also these attributes:
     \begin{itemize}
       \item \xmlAttr{R0values}, \xmlDesc{float, required field}, interval of values (comma separated values)
          that implies a reliability value equal to $0$ for the input variable
       \item \xmlAttr{R1values}, \xmlDesc{float, required field}, interval of values (comma separated values)
          that implies a reliability value equal to $1$ for the input variable
     \end{itemize}
   \item \xmlNode{target}, \xmlDesc{string, required field}, ID of the output variable. This nodes needs to
      contain also the attribute \xmlAttr{values}, \xmlDesc{string, required field}, interval of
      values of the output target variable that defines the desired outcome
\end{itemize}

\textbf{Example:}
This example shows an example where it is desired to calculate all available risk importance
measures for two input variables (i.e., pumpTime and valveTime)
given an output target variable (i.e., Tmax).
A value of the input variable pumpTime in the interval $[0,240]$ implies a reliability
value of the input variable pumpTime equal to $0$.
A value of the input variable valveTime in the interval $[0,60]$ implies a reliability
value of the input variable valveTime equal to $0$.
A value of the input variables valveTime and pumpTime in the interval $[1441,2880]$ implies a
reliability value of the input variables equal to $1$.
The desired outcome of the output variable Tmax occurs in the interval $[2200,2500]$.
\begin{lstlisting}[style=XML,morekeywords={subType,debug,name,class,type}]
<Simulation>
  ...
  <Models>
    ...
    <PostProcessor name="riskMeasuresDiscrete" subType="RiskMeasuresDiscrete">
      <measures>B,FV,RAW,RRW</measures>
      <variable R0values='0,240' R1values='1441,2880'>pumpTime</variable>
      <variable R0values='0,60'  R1values='1441,2880'>valveTime</variable>
      <target   values='2200,2500'                  >Tmax</target>
    </PostProcessor>
    ...
  </Models>
  ...
</Simulation>
\end{lstlisting}

This Post-Processor allows the user to consider also multiple datasets (a data set for each initiating event)
and calculate the global risk importance measures.
This can be performed by:
\begin{itemize}
  \item Including all datasets in the step
\begin{lstlisting}[style=XML,morekeywords={subType,debug,name,class,type}]
<Simulation>
  ...
  </Steps>
    ...
    <PostProcess name="PP">
      <Input   class="DataObjects"  type="PointSet"        >outRun1</Input>
      <Input   class="DataObjects"  type="PointSet"        >outRun2</Input>
      <Model   class="Models"       type="PostProcessor"   >riskMeasuresDiscrete</Model>
      <Output  class="DataObjects"  type="PointSet"        >outPPS</Output>
      <Output  class="OutStreams"   type="Print"           >PrintPPS_dump</Output>
    </PostProcess>
  </Steps>
  ...
</Simulation>
\end{lstlisting}
  \item Adding in the Post-processor the frequency of the initiating event associated to each dataset
\begin{lstlisting}[style=XML,morekeywords={subType,debug,name,class,type}]
<Simulation>
  ...
  <Models>
    ...
    <PostProcessor name="riskMeasuresDiscrete" subType="SR2ML.RiskMeasuresDiscrete">
      <measures>FV,RAW</measures>
      <variable R1values='-0.1,0.1' R0values='0.9,1.1'>Astatus</variable>
      <variable R1values='-0.1,0.1' R0values='0.9,1.1'>Bstatus</variable>
      <variable R1values='-0.1,0.1' R0values='0.9,1.1'>Cstatus</variable>
      <variable R1values='-0.1,0.1' R0values='0.9,1.1'>Dstatus</variable>
      <target   values='0.9,1.1'>outcome</target>
      <data     freq='0.01'>outRun1</data>
      <data     freq='0.02'>outRun2</data>
    </PostProcessor>
    ...
  </Models>
  ...
</Simulation>
\end{lstlisting}

\end{itemize}

This post-processor can be made time dependent if a single HistorySet is provided among the other data objects.
The HistorySet contains the temporal profiles of a subset of the input variables. This temporal profile can be only
boolean, i.e., 0 (component offline) or 1 (component online).
Note that the provided history set must contains a single History; multiple Histories are not allowed.
When this post-processor is in a dynamic configuration (i.e., time-dependent), the user is required to specify an xml
node \xmlNode{temporalID} that indicates the ID of the temporal variable.
For each time instant, this post-processor determines the temporal profiles of the desired risk importance measures.
Thus, in this case, an HistorySet must be chosen as an output data object.
An example is shown below:
\begin{lstlisting}[style=XML,morekeywords={subType,debug,name,class,type}]
<Simulation>
  ...
  <Models>
    ...
    <PostProcessor name="riskMeasuresDiscrete" subType="SR2ML.RiskMeasuresDiscrete">
      <measures>B,FV,RAW,RRW,R0</measures>
      <variable R1values='-0.1,0.1' R0values='0.9,1.1'>Astatus</variable>
      <variable R1values='-0.1,0.1' R0values='0.9,1.1'>Bstatus</variable>
      <variable R1values='-0.1,0.1' R0values='0.9,1.1'>Cstatus</variable>
      <target   values='0.9,1.1'>outcome</target>
      <data     freq='1.0'>outRun1</data>
      <temporalID>time</temporalID>
    </PostProcessor>
    ...
  </Models>
  ...
  <Steps>
    ...
    <PostProcess name="PP">
      <Input     class="DataObjects"  type="PointSet"        >outRun1</Input>
      <Input     class="DataObjects"  type="HistorySet"      >timeDepProfiles</Input>
      <Model     class="Models"       type="PostProcessor"   >riskMeasuresDiscrete</Model>
      <Output    class="DataObjects"  type="HistorySet"      >outHS</Output>
      <Output    class="OutStreams"   type="Print"           >PrintHS</Output>
    </PostProcess>
    ...
  </Steps>
  ...
</Simulation>
\end{lstlisting}

    \section{Margin Models}
\label{sec:MarginModels}

\textbf{Margin Models} are models designed to calculate the margin of a component for margin-based reliability 
calculations.

The classes of models considered here are as follows:
\begin{itemize}
	\item PointSetMarginModel, i.e. model \xmlAttr{type} is \xmlString{PointSetMarginModel}
\end{itemize}

The specifications of these models must be defined within a RAVEN \xmlNode{ExternalModel}. This
XML node accepts the following attributes:
\begin{itemize}
	\item \xmlAttr{name}, \xmlDesc{required string attribute}, user-defined identifier of this model.
	\nb As with other objects, this identifier can be used to reference this specific entity from other
	input blocks in the XML.
	\item \xmlAttr{subType}, \xmlDesc{required string attribute}, defines which of the subtypes should
	be used. For margin models, the user must use \xmlString{SR2ML.MarginModel} as subtype.
\end{itemize}

In the margin \xmlNode{ExternalModel} input block, the following XML subnodes are required:
\begin{itemize}
	\item \xmlNode{variable}, \xmlDesc{string, required parameter}. Comma-separated list of variable
	names. Each variable name needs to match a variable used or defined in the matign model or variable
	coming from other RAVEN entities (i.e., Samplers, DataObjects, and Models).
	\item \xmlNode{MarginModel}, \xmlDesc{required parameter}. The node is used to define the maintenance
	model, and it contains the following required XML attribute:
	\begin{itemize}
		\item \xmlAttr{type}, \xmlDesc{required string attribute}, user-defined identifier of the margin model.
	\end{itemize}
\end{itemize}

In addition, if the user wants to use the \textbf{alias} system, the following XML block can be input:
\begin{itemize}
	\item \xmlNode{alias} \xmlDesc{string, optional field} specifies alias for
	any variable of interest in the input or output space for the ExternalModel.
	%
	These aliases can be used anywhere in the RAVEN input to refer to the ExternalModel
	variables.
	%
	In the body of this node, the user specifies the name of the variable that the ExternalModel is
	going to use (during its execution).
	%
	The actual alias, usable throughout the RAVEN input, is instead defined in the
	\xmlAttr{variable} attribute of this tag.
	\\The user can specify aliases for both the input and the output space. As a sanity check, RAVEN
	requires an additional required attribute \xmlAttr{type}. This attribute can be either ``input'' or ``output.''
	%
	\nb The user can specify as many aliases as needed.
	%
	\default{None}
\end{itemize}

\subsection{PointSetMarginModel Model}
For the PointSetMarginModel model, in the margin \xmlNode{MarginModel} input block, the following XML subnodes 
are required:
\begin{itemize}
	\item \xmlNode{failedDataFileID}, \xmlDesc{string, required parameter}. ID of the file containing the failure data
	      to be used as reference data
	\item \xmlNode{marginID}, \xmlDesc{required parameter}. ID of the variable where the margin needs to be stored
	\item \xmlNode{map}, \xmlDesc{required parameter}. This node is used to map actual measured data dimension and the 
	      failure data dimension. In here, the ID of the Raven variable representing a dimension of the actual data
	\begin{itemize}
		\item \xmlAttr{var}, \xmlDesc{required string attribute}, the ID of the corresponding dimension in the failure data.
	\end{itemize}
\end{itemize}

Example XML:
\begin{lstlisting}[style=XML]
    <ExternalModel name="PointSetMargin" subType="SR2ML.MarginModel">
      <variables>actualTime,actualTemp,marginPS1</variables>
      <MarginModel type="PointSetMarginModel">
        <failedDataFileID>failureData.csv</failedDataFileID>
        <marginID>marginPS1</marginID>
	      <map var='time'>actualTime</map>
	      <map var='avgT'>actualTemp</map>
      </MarginModel>
    </ExternalModel>
\end{lstlisting}

    \section*{Document Version Information}
    This document has been compiled using the following version of the plug-in git repository:
    \newline
    \input{../version.tex}

    % ---------------------------------------------------------------------- %
    % References
    %
    \clearpage
    % If hyperref is included, then \phantomsection is already defined.
    % If not, we need to define it.
    \providecommand*{\phantomsection}{}
    \phantomsection
    \addcontentsline{toc}{section}{References}
    \bibliographystyle{ieeetr}
    \bibliography{user_manual}


    % ---------------------------------------------------------------------- %

\end{document}
